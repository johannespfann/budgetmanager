
\section{Codeconvension}
\subsection{Styles}
Styles stehen für Elemente die in einer Anwendung einheitlich sein sollen. Die Benennung der Styles wird einmal unterteilt durch die Dateibenamung und dem Style-Element.
\paragraph{Style - Dateinamen}
In einer Datei werden alle Style-Elemente eines Typs definiert. Es gibt eine globale Style Datei (In dieser sammeln sich alle Styles, die noch keine Zuordnung besitzen oder globale Bedeutung besitzen wie z.B Schriftarten/Schriftfarben.

\begin{itemize}
\item \{style\}\_\{ElementKlasse\}\_\{Elementaufgabe/ort\}\_\{..\}
\end{itemize}

Beispiel:
\begin{itemize}
\item style\_textview\_global;
\item style\_textview\_resultfield;
\item style\_button\_global;
\item style\_button\_keyboardbutton\_defaultdigit;
\item style\_button\_keyboardbutton\_return;
\item style\_button\_menuebutton\_default;
\end{itemize}


\paragraph{Stylename}
\begin{itemize}
\item \{ElementKlasse\}\_\{Elementaufgabe/ort\}\_\{...\}
\item textview\_global
\item textview\_result;
\item button\_global
\item button\_keyboardbutton
\end{itemize}


\subsection{Text}

Die Texte sind meist für eine bestimmte Ansicht wichtig. Allerdings gibt es auch globale Texte wie z.B. Zurück, Bestätigen usw.

\paragraph{Text-Datei}

\begin{itemize}
	\item strings\_\{Elementort\}\_\{Elementklasse\}\_\{...\}
\end{itemize}

\begin{itemize}
\item global\_textview
\item mainpage\_textview
\item mainpage\_button
\end{itemize}

\paragraph{Textname}

\begin{itemize}
	\item \{Elementort\}\_\{Elementklasse\}\_\{text\}
\end{itemize} 

\begin{itemize}
	\item global\_textview
	\item mainpage\_textview\_text
	\item mainpage\_textview\_singleplayer
	\item mainpage\_textview\_multiplayer
	\item mainpage\_textview\_rangliste
	
\end{itemize}

\subsection{Git Commitmessage-convention}

\begin{table}[H]
\begin{tabular}{|l|p{5cm}|p{9cm}|}
\hline
\textbf{Begriff} & \textbf{Zweck} & \textbf{Beispiel} \\
Initial import & Erstmaliges Einchecken eines Projekts & - Initial import \\
\hline
Added          & Hinzufügen neuer Funktionalität &  - Added support for 
														EJB injection in BusinessObjects\\
\hline
Removed        & Entfernen bestehender Funktionalität & - Removed 
														deprecated method 
														getAllThatUglyStuff() \\
\hline
\end{tabular}
\end{table}

