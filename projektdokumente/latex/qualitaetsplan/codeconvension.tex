
\section{Codeconvension}
\subsection{Styles}
Styles stehen für Elemente die in einer Anwendung einheitlich sein sollen. Die Benennung der Styles wird einmal unterteilt durch die Dateibenamung und dem Style-Element.
\paragraph{Style - Dateinamen}
In einer Datei werden alle Style-Elemente eines Typs definiert. Es gibt eine globale Style Datei (In dieser sammeln sich alle Styles, die noch keine Zuordnung besitzen oder globale Bedeutung besitzen wie z.B Schriftarten/Schriftfarben.

\begin{itemize}
\item \{style\}\_\{ElementKlasse\}\_\{Elementaufgabe/ort\}\_\{..\}
\end{itemize}

Beispiel:
\begin{itemize}
\item style\_textview\_global;
\item style\_textview\_resultfield;
\item style\_button\_global;
\item style\_button\_keyboardbutton\_defaultdigit;
\item style\_button\_keyboardbutton\_return;
\item style\_button\_menuebutton\_default;
\end{itemize}


\paragraph{Stylename}
\begin{itemize}
\item \{ElementKlasse\}\_\{Elementaufgabe/ort\}\_\{...\}
\item textview\_global
\item textview\_result;
\item button\_global
\item button\_keyboardbutton
\end{itemize}


\subsection{Text}

Die Texte sind meist für eine bestimmte Ansicht wichtig. Allerdings gibt es auch globale Texte wie z.B. Zurück, Bestätigen usw.

\paragraph{Text-Datei}

\begin{itemize}
	\item \{Elementort\}\_\{Elementklasse\}\_\{...\}
\end{itemize}

\begin{itemize}
\item global\_textview\_confirm
\item mainpage\_textview\_header 
\item mainpage\_button\_singleplayer 
\end{itemize}

\paragraph{Textname}

\begin{itemize}
	\item \{Elementort\}\_\{text\}
\end{itemize} 

\begin{itemize}
	\item global\_textview
	\item mainpage\_text
\end{itemize}